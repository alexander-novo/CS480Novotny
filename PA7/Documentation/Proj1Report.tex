% This tells the program how to initially format my document.
\documentclass[twoside]{article}

% List of packages to be included for the use of special commands and symbols.
\usepackage{caption}
\usepackage{graphicx}
\usepackage{qtree}
\usepackage[letterpaper]{geometry}
\usepackage{tasks} % For multi-column lists


% Automatically inserts and formats a title page for the document, to be used when the \maketitle command is entered.
\title{Solar System}
\date{October 20, 2017}
\author{Alexander Novotny \and Matthew Johnson}


% This is where the bulk of the document begins.
\begin{document}

% This makes it so that the title page does not have page numbering.
\pagenumbering{gobble}
\maketitle

\newpage

% Blank page after title page for double-sided printing
\shipout\null

% This is where I designate the creation of my Table of Contents.
\tableofcontents

\newpage

% Blank page after toc for double-sided printing
\shipout\null

\pagenumbering{arabic}

\section{Tech Manual}

\subsection{Dependencies}

This project requires GLEW, GLM, SDL, Assimp, and ImageMagick. These libraries can be installed using \texttt{sudo apt-get install libglew-dev libglm-dev libsdl2-dev libassimp-dev libmagick++-dev} on Ubuntu.

This project also uses ImGUI and Json for Modern C++, but the headers for those are already included

\subsection{Building and Running}

To build the project, cmake is required. Simply use the \texttt{cmake} command to generate a makefile, and then the \texttt{make} command to build the project. If this is successful, there are several ways to run the program:


\begin{tasks}[counter-format=\ ](1)
	\task \texttt{Tutorial} - Run the program with the default config.json file
	\task \texttt{Tutorial --help} - Pull up the help menu
	\task \texttt{Tutorial <config>} - Run the program with the specififed config file
\end{tasks}

\subsection{Controls}

\begin{tasks}[counter-format=\ ](1)
	\task WASD - Detach the camera from any planet and move it in the cardinal directions
	\task Space - Move the camera upwards
	\task Shift - Move the camera downwards
	\task left/right arrows - Rotate the camera
	\task CLick and Drag to rotate the camera around a planet
	\task Scroll wheel to zome in and out on a planet
\end{tasks}

\newpage

\subsection{Menu}

\begin{figure}[h]
\centering
\includegraphics{menu.png}
\end{figure}

\begin{tasks}[counter-format=\ ](1)
	\task \textbf{Camera Distance} - Change how far away the camera is from a planet
	\task \textbf{Camera Rotation} - Change where the camera looks at the planet from
	\task \textbf{Realistic/Close Scale} - Make planets have correctly scaled sizes and distances, or easily viewable
	\task \textbf{Orbits} - Draw orbit lines
	\task \textbf{Labels} - Label planet/moon names
\end{tasks}

\subsubsection{Planet Controls}

\begin{tasks}[counter-format=\ ](1)
	\task \textbf{Select Satellite} - Center Camera on a specific planet/moon, and choose which planet the controls affect
	\task \textbf{Time Scale} - Simulation Speed for this planet and all of its satellites
	\task \textbf{Orbit/Spin Speed} - Change how fast a plant orbits or rotates
	\task \textbf{Orbit/Spin Direction} - Change the orbit/spin direction of a planet
\end{tasks}

\subsubsection{Experimental Features}

\begin{tasks}[counter-format=\ ](1)
	\task \textbf{Draw Shadows} - Make planets and moons cast shadows on each other
\end{tasks}

\newpage

\subsection{Extra Credit}

\begin{tasks}[counter-format=tsk[1])](1)
	\task Menu System
	\task Configuration File
	\task Live Adjustment of Simulation Speed (see section 1.4)
	\task Planet Orbit Paths
	\task Scaled/Realistic View
	\task Planet Rings
	\task Specular/Normal Maps on Earth
	\task Earth nighttime textures
	\task Shadow Mapping
	\task Tilted Orbits
	\task Skybox
\end{tasks}

\newpage

\section{Showcase}

\begin{figure}[h]
\centering
\caption{The tilted orbits of Uranus' moons and rings}
\includegraphics[width=0.79\textwidth]{uranus.png}
\end{figure}

\begin{figure}[h]
\centering
\caption{A view of The Sun and Mars from Mars' moon Deimos}
\includegraphics[width=0.79\textwidth]{deimos.png}
\end{figure}

\newpage

\begin{figure}[h]
\centering
\captionsetup{justification=centering}
\caption{Earth's normal map makes the Andes cast a shadow over Western South America, causing their mornings to be later}
\includegraphics[width=0.79\textwidth]{nightEarth.png}
\end{figure}

\newpage

\begin{figure}[h]
\centering
\captionsetup{justification=centering}
\caption{The Sun shines on the Red Sea}
\includegraphics[width=0.79\textwidth]{dayEarth.png}
\end{figure}

\newpage

\section{Report}

\subsection{Class Diagram}

\Tree[.Engine [.Window ]
              [.Graphics [.Object [.Shader ]
                                  [.Model ]
                                  [.Texture ] ] ]
              [.Menu ] ]

\end{document}